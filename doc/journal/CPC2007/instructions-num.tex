\documentclass{elsart}
\usepackage{ifpdf}
\usepackage{graphicx,amssymb,lineno}
\ifpdf
\usepackage[%
  pdftitle={Instructions for use of the document class
    elsart},%
  pdfauthor={Simon Pepping},%
  pdfsubject={The preprint document class elsart},%
  pdfkeywords={instructions for use, elsart, document class},%
  pdfstartview=FitH,%
  bookmarks=true,%
  bookmarksopen=true,%
  breaklinks=true,%
  colorlinks=true,%
  linkcolor=blue,anchorcolor=blue,%
  citecolor=blue,filecolor=blue,%
  menucolor=blue,pagecolor=blue,%
  urlcolor=blue]{hyperref}
\else
\usepackage[%
  breaklinks=true,%
  colorlinks=true,%
  linkcolor=blue,anchorcolor=blue,%
  citecolor=blue,filecolor=blue,%
  menucolor=blue,pagecolor=blue,%
  urlcolor=blue]{hyperref}
\fi

\renewcommand\floatpagefraction{.2}
\makeatletter
\def\elsartstyle{%
    \def\normalsize{\@setfontsize\normalsize\@xiipt{14.5}}
    \def\small{\@setfontsize\small\@xipt{13.6}}
    \let\footnotesize=\small
    \def\large{\@setfontsize\large\@xivpt{18}}
    \def\Large{\@setfontsize\Large\@xviipt{22}}
    \skip\@mpfootins = 18\p@ \@plus 2\p@
    \normalsize
}
\@ifundefined{square}{}{\let\Box\square}
\makeatother

\def\file#1{\texttt{#1}}

\pagestyle{plain}
\begin{document}

\begin{frontmatter}
\title{Instructions for use\\of the document class \file{elsart}}

\author{Simon Pepping}
\address{Elsevier, P.O. Box 103, 1000 AC Amsterdam,
Netherlands}

\ead{s.pepping@elsevier.com}
\ead[url]{authors.elsevier.com/locate/latex}

\begin{abstract}
This article discusses several features of preparing articles
with the \file{elsart} document style, using numbered style bibliographic
references.
\end{abstract}

\begin{keyword}
\file{elsart}, document class, instructions for use
\PACS 01.30.$-$y
\end{keyword}
\end{frontmatter}

\section{Introduction}
\label{intro}

This article discusses how to prepare articles with the \file{elsart}
document class.  For more general information about \LaTeX{}, see the
\LaTeX{} manual written by Lamport \cite{Lamp86}.

Elsevier has prepared the following \LaTeX{} support files for
authors:
\begin{itemize}
\item The document class \file{elsart.cls}, which provides a preprint
  layout.
\item The document classes \file{elsart1p.cls}, \file{elsart3p.cls},
  \file{elsart5p.cls}, which each provide a layout in one of
  Elsevier's standard journal styles, called 1+, 3+ and 5+.
\item The instructions for use of \file{elsart},
  \file{instructions-harv} for use with a Harvard-style bibliography,
  and \file{instructions-num} for use with a numbered style
  bibliography.
\item Template files for a quick start of your \LaTeX{} article with
  \file{elsart},
  \newline
  \file{template-harv.tex} for use with a Harvard-style
  bibliography, and
  \newline
  \file{template-num.tex} for use with a numbered
  style bibliography.
\item Styles for BibTeX, \file{elsart-harv.bst} for a Harvard-style
  bibliography, and \file{elsart-num.bst} for a numbered style
  bibliography.
\end{itemize}
The files are freely available from Elsevier's Author Gateway
\url{http://authors.elsevier.com/locate/latex}. 

On Elsevier's Author Gateway you will also find support files for CRC
journal articles. Support for monographs or contributed book chapters
may be obtained via the publisher of the book.

Elsevier's \LaTeX{} support files can also be obtained from one of the
servers of the Comprehensive TeX Archive Network (CTAN) in the
directory\newline
\href{ftp://ctan.tug.org/tex-archive/macros/latex/contrib/supported/elsevier}%
{\texttt{/tex-archive/macros/latex/contrib/supported/elsevier}}. CTAN is a
mirrored network of FTP servers, with the following web front ends:
\href{http://www.tex.ac.uk}{\texttt{www.tex.ac.uk}},
\href{http://www.dante.de/software/ctan}{\texttt{www.dante.de/software/ctan}}
(in German) and \href{http://www.ctan.org}{\texttt{www.ctan.org}}. The
network is widely mirrored, see
\url{http://www.tug.org/tex-archive/CTAN.sites}. It holds up-to-date
copies of all the public-domain versions of \TeX, \LaTeX, Metafont and
ancillary programs.

Note that CTAN is not related to Elsevier, and that Elsevier's author
support cannot accept complaints or answer questions about the
availability of any CTAN server.

The non-Elsevier macro packages recommended later in this document and
many other useful macro packages can also be obtained from CTAN.

In the following sections we show how you may use the \file{elsart}
document class.

\section{Options}

The \file{elsart} document class enables the following options:

\begin{description}
  
\item[doublespacing, reviewcopy] This is a single option with two
  names to obtain double line spacing, as is sometimes required for
  copies submitted for review.
  
\item[seceqn, secthm] The option \texttt{seceqn} numbers the equation
  environments per section. The option \texttt{secthm} does the same
  for the \texttt{thm} environment. In elsart all predefined theorem
  environments except Algorithm, Note, Summary and Case use the same
  counter as the \texttt{thm} environment.

\item[draft, final] As in many other document classes, these are
  options to produce draft and final layout. In the draft layout you
  will see warnings for overfull boxes. You also need draft layout to
  test your formulas on a narrower display width, see option
  \texttt{narrowdisplay}.
  
\item[narrowdisplay] Many Elsevier journals print their text in two
  columns. Because the preprint layout uses a larger line width
  than such columns, the formulas are too wide for the line width in
  print. In draft mode (see the \texttt{draft} option) you can use the
  \texttt{narrowdisplay} option to force a narrower width for
  displayed formulas. The width is roughly equal to the column width
  of the printed journals, compensated for the larger font size of the
  preprint layout. The \texttt{narrowdisplay} option is ineffective
  with packages which redefine the equation environments, such as
  \texttt{amsmath}.
  
  The \texttt{narrowdisplay} option is especially useful for journals
  for which the articles are printed from the author's \LaTeX{} file.
  This is the case for a number of mathematics journals. When you
  break your formulas such that they fit in the narrow column width,
  the typesetter will be able to retain most of your breaks. Article
  for other journals are printed after transformation to an XML file.
  For such journals the formula layout in the \LaTeX{} file is always
  lost in the transformation.
  
  The narrow display width is obtained by giving the formulas a larger
  indent. Too wide formulas in the one-line display environments
  \texttt{equation} and \texttt{displaymath} will show an overfull
  rule:
  
  \def\testformula{%
    \sum_{i=0}^{\infty}A^n\int \mathrm{d}x\, \frac{F_n(x)}{A_n + B_n} =
    B^n C^n \int\mathrm{d}x\,\int \mathrm{d}y\,
    \frac{G_n(x,y)}{\mathcal{A}_n{x} + \mathcal{B}_n{y}}
    }
  \overfullrule 5pt
  \mathindent\linewidth\relax
  \advance\mathindent-259pt
  \begin{equation}
    \label{eq:1}
    \testformula.
  \end{equation}

  This is not the case for the multiline display environment
  \texttt{eqnarray}. But in both cases too wide formulas will extend
  into the right margin, giving you visual feedback:

  \begin{eqnarray}
    \label{eq:2}
    \testformula,\nonumber\\
    \testformula.
  \end{eqnarray}

  When you switch off draft mode, the formulas will have their normal
  indentation, and too wide formulas will no longer be signalled.

\end{description}

\section{Print layout}
\label{printlayout}

Elsevier also makes available a few document classes that roughly
reproduce the layout of the printed journals. The majority of Elsevier
journals use one of a small set of standard layouts. We have document
classes for three of those layouts:

\begin{description}
\item[elsart1p] text width 32 picas (134 mm), text height 47 lines,
  one column.
\item[elsart3p] text width 39 picas (164 mm), text height 51 lines,
  one or two columns.
\item[elsart5p] text width 43.5 picas (183 mm), text height 57 lines,
  two columns.
\end{description}

These classes can be used in the same way as elsart. If you prepared
an article for elsart, you can run it with one of these print layout
classes without changes to the markup. In fact, they use elsart and
you must have elsart on your system as well.

Note that the layout is only roughly the same as that of the printed
journal. One major source of differences is the font. The printer uses
a different font with different character widths, which will cause
deviations in layout. There are various other sources of small
differences. You cannot use the layout of one of these classes to make
claims on the final layout of your article.

\section{Frontmatter}
\label{frontmatter}

The \texttt{elsart} document class has a separate 
\texttt{frontmatter} environment for the title, authors, addresses,
abstract and keywords.
\begin{itemize}
\item \verb|\title|: As in standard \LaTeX, e.g. 
\verb|\title{Model}|.
\item \verb|\author|: Different from standard \LaTeX, the \verb|\author|
command contains only one author and no address.
Multiple authors go into multiple \verb|\author| commands,
separated from each other by commas.
The address goes into a separate 
\verb|\address| command.
Example: \verb|\author{D.E. Knuth}|.
\item \verb|\address|: Here goes the address, e.g. 
\verb|\address{CERN, Geneva}|.
\item \verb|\thanks| and \verb|\thanksref|: 
These provide footnotes to the title, authors and addresses. 
The \verb|\thanksref| command takes a label: \verb|\thanksref{label}|
to relate it to the \verb|\thanks| command with 
the same label: \verb|\thanks[label]|. There can be several
references to a single \verb|\thanks| command. Example:\\
\verb|\title{Model\thanksref{titlefn}}| and\\
\verb|\thanks[titlefn]{Supported by grants.}|
\item \verb|\corauth| and \verb|\corauthref|: 
These provide footnotes to mark the corresponding author and the
correspondence address. They are used in the same manner as
\verb|\thanks| and \verb|\thanksref|. Example:\\
\verb|\author{A. Name\corauthref{cor}}| and\\
\verb|\corauth[cor]{Corresponding author. Address: ... .}|
\item \verb|\ead|:
This command should be used for the email address or the URL of the
author. It refers to the `current author', i.e., the author last
mentioned before the command.
When it holds a URL, this should be indicated by setting the
optional argument to `url'. Example:
\verb|\ead{s.pepping@elsevier.com}|,
\verb|\ead[url]{authors.elsevier.com/locate/latex}|.
\end{itemize}

It is not necessary to give a \verb|\maketitle| command. The title,
authors and addresses are printed as soon as \TeX{} sees them.

The authors and addresses can be combined in one of two ways:
\begin{itemize}
\item The simplest way lists the authors of one address or one group
  of addresses, followed by the address or addresses, and so on for
  all addresses or groups of addresses.
\item The other way first lists all authors, and then all addresses.
The authors and addresses are related to each other by labels:
\verb|\author[label1]{Name1}| corresponds to
\verb|\address[label1]{Address1}|. Example:
\begin{verbatim}
\author[South]{T.R. Marsh},
\author[Oxford]{S.R. Duck}
\address[South]{University of Southampton, UK}
\address[Oxford]{University of Oxford, UK}
\end{verbatim}
\end{itemize}

See the extensive examples in figs.
\ref{ex:explinput}, \ref{ex:exploutput}, 
\ref{ex:implinput}, \ref{ex:imploutput}.

If you put the frontmatter in an included file, that file should
contain the whole frontmatter, including its \texttt{begin} and
\texttt{end} commands. Otherwise, the labels of the frontmatter will
remain undefined.

%%%%%%%%%%%%%%%%%%%%%%%%%%%%%%ex:explinput%%%%%%%%%%%%%%%%%%%%%%%%%%%%%%

\begin{figure}[p]

\caption{Article opening with explicit links (input)} \label{ex:explinput}
\vspace{1pc}

\begin{verbatim}
\documentclass{elsart}
\usepackage{graphicx,amssymb}
\journal{New Astronomy}
\begin{document}
\begin{frontmatter}

\title{Stroboscopic Doppler tomography of FO Aqr}
\author[South]{T.R. Marsh\corauthref{cor}},
\corauth[cor]{Corresponding author.}
\ead{trm@astro.soton.ac.uk}

\author[Oxford]{S.R. Duck\thanksref{now}}
\thanks[now]{Present address: Systems Engineering and Assessment Ltd.,
Beckington Castle, PO Box 800, Bath BA3 6TB, UK.}
\ead{srd@sea.co.uk}

\address[South]{University of Southampton, Department of Physics,
Highfield, Southampton SO17 1BJ, UK} 
\address[Oxford]{University of Oxford, Department of Physics, Nuclear
Physics Laboratory, Keble Road, Oxford, OX1 3RH, UK}
 
\begin{abstract} 
FO Aqr is a close binary star in
which a magnetic white dwarf accretes from a cool companion. Light
curves and spectra show variations on the orbital frequency, the
white dwarf's spin frequency and combinations of the two.
\end{abstract}
\begin{keyword}
Accretion, accretion disks \sep Line: profiles \sep
Binaries: close \sep Novae, cataclysmic variables 
\PACS 97.10.Gz \sep 97.30.Qt \sep 97.80.Gm 
\end{keyword}
\end{frontmatter}

\section{Introduction}
FO Aqr is a member of the DQ~Her class of stars which
are close binary stars in which a magnetic white dwarf accretes from
a late-type main-sequence secondary star. These stars have most
recently been reviewed in Ref. \cite{Patterson94}. 
\end{verbatim}

\end{figure}

%%%%%%%%%%%%%%%%%%%%%%%%%%%%%ex:exploutput%%%%%%%%%%%%%%%%%%%%%%%%%%%%%

\begin{figure}[p]
\caption{Article opening with explicit links (output)\label{ex:exploutput}} 
\vspace{1pc}

\begin{minipage}{\textwidth}
\elsartstyle
\parskip 12pt
%
\renewcommand{\thempfootnote}{\fnsymbol{mpfootnote}}
%
\leftskip=2pc
%
\begin{center}
{\LARGE Stroboscopic Doppler tomography of FO Aqr}\\[30pt]
%
\large
T.R. Marsh$^{\mathrm{a},*}$,
S.R. Duck$^{\mathrm{b},1}$\\[12pt]
%
\small\itshape
$^a$University of Southampton, Department of Physics,\\
Highfield, Southampton SO17 1BJ, UK\\
$^b$University of Oxford, Department of Physics, Nuclear Physics\\
Laboratory, Keble Road, Oxford, OX1 3RH, UK
 
\end{center}

\bigskip
\leftskip=0pt

\hrule\vskip 8pt
\begin{small}
{\bfseries Abstract}
\parindent 1em

FO Aqr is a close binary star in
which a magnetic white dwarf accretes from a cool companion. Light
curves and spectra show variations on the orbital frequency, the
white dwarf's spin frequency and combinations of the two.

\noindent\textit{Key words:}
Accretion, accretion disks, Line: profiles,
Binaries: close, Novae, cataclysmic variables\\
\textit{PACS:} 97.10.Gz, 97.30.Qt, 97.80.Gm 
\end{small}
%
\vskip 10pt\hrule 

\leftskip=0pt

\vspace{24pt}

\textbf{Introduction}

FO Aqr is a member of the DQ~Her class of stars which
are close binary stars in which a magnetic white dwarf accretes from
a late-type main-sequence secondary star. These stars have most
recently been reviewed in Ref. [1]. 

\begin{footnotesize}
\leavevmode
\rlap{$^*$}\hspace{1pc}Corresponding author.\\
\rlap{$^1$}\hspace{1pc}Present address: Systems Engineering and
Assessment Ltd., Beckington Castle, PO Box 800, Bath BA3 6TB, UK.\\
\rlap{}\hspace{1pc}\textit{Email addresses:} trm@astro.soton.ac.uk
(T.R. Marsh), srd@sea.co.uk (S.R. Duck).
\end{footnotesize}

\leavevmode\hbox to \hsize{\small\slshape Preprint submitted to New Astronomy
\hfil 21 August 1997}

\end{minipage}

\bigskip

\end{figure}

%%%%%%%%%%%%%%%%%%%%%%%%%%%%%%ex:implinput%%%%%%%%%%%%%%%%%%%%%%%%%%%%%%

\begin{figure}[p]
\caption{Article opening with implicit links (input)\label{ex:implinput}}
\vspace{1pc}

\begin{verbatim}
\documentclass{elsart}

\begin{document}
\begin{frontmatter}
\title{Integrability in
       random matrix models\thanksref{talk}}
\thanks[talk]{Expanded version of a talk
  presented at the Singapore Meeting on
  Particle Physics (Singapore, August 1990).}

\author{L. Alvarez-Gaum\'{e}\corauthref{cor}}
\address{Theory Division, CERN,
  CH-1211 Geneva 23, Switzerland}
\corauth[cor]{Corresponding author.}
\ead{lag@cern.ch}

\author{C. Gomez\corauthref{cor}\thanksref{SNSF}}
\address{D\'{e}partment de Physique Th\'{e}orique, 
  Universit\'{e} de Gen\`{e}ve,
  CH-1211 Geneva 4, Switzerland}
\ead{cg@ug.ch}

\author{J. Lacki}
\address{School of Natural Sciences,
  Institute for Advanced Study,
  Princeton, NJ 08540, USA}
\ead[url]{www.ias.edu/~jl}
\thanks[SNSF]{Supported by the
  Swiss National Science Foundation}

\begin{abstract}
We prove the equivalence between the recent matrix
model formulation of 2D gravity and lattice
integrable models.  For even potentials this
system is the Volterra hierarchy.
\end{abstract}
\end{frontmatter}

\section{Introduction}
Some aspects of the recently discovered
non-perturbative solutions to non-critical strings
\cite{Patterson94} can be better understood and
clarified directly in terms of the integrability
properties of the random matrix model.
...
\end{verbatim}

\end{figure}

%%%%%%%%%%%%%%%%%%%%%%%%%%%%%ex:imploutput%%%%%%%%%%%%%%%%%%%%%%%%%%%%%

\begin{figure}[p]
\caption{Article opening with implicit links (output)\label{ex:imploutput}}
\vspace{1pc}

\begin{minipage}{\textwidth}
\elsartstyle
%
\renewcommand{\thempfootnote}{\fnstar{mpfootnote}}
%
\leftskip=2pc
%
\begin{center}
{\LARGE Integrability in random matrix models$^\star$}\\[30pt]
\footnotetext[1]{\upshape Expanded version of a talk presented at the
Singapore Meeting on Particle Physics (Singapore, August 1990).}
%
\large
L. Alvarez-Gaum\'{e}$^*$         \\[6pt]
{\small\itshape Theory Division, CERN,
  CH-1211 Geneva 23, Switzerland}\\[18pt]
C. Gomez$^{*,1}$                   \\[6pt]
{\small\itshape D\'{e}partment de Physique Th\'{e}orique, 
  Universit\'{e} de Gen\`{e}ve,
  CH-1211 Geneva 4, Switzerland} \\[15pt]
J. Lacki                         \\[6pt]
{\small\itshape School of Natural Sciences, Institute for Advanced Study,
  Princeton, NJ 08540, USA}
% BW:
\end{center}
%
\renewcommand{\thempfootnote}{\astsymbol{mpfootnote}}
\footnotetext[1]{Corresponding author.}
\setbox0=\hbox{\footnotesize 1}
\edef\thempfootnote{\hskip\wd0}
\footnotetext[0]{\textit{Email adresses:} lag@cern.ch
  (L. Alvarez-Gaum\'{e}), cg@ug.ch (C. Gomez).}
\footnotetext[1]{\textit{URL:} www.ias.edu/\~{ }jl (J. Lacki).}
\renewcommand{\thempfootnote}{\arabic{mpfootnote}}
\footnotetext[1]{Supported by the Swiss National Science Foundation}

\bigskip
\leftskip=0pt

\hrule\vskip 8pt
\begin{small}
{\bfseries Abstract}
\parindent 1em

We prove the equivalence between the recent matrix model
formulation of 2D gravity and lattice integrable models. For even
potentials this system is the Volterra hierarchy.\par
\end{small}
\vskip 10pt\hrule 

\medskip

\section*{1. Introduction}

Some aspects of the recently discovered non-perturbative
solutions to non-critical strings [1] can be better
understood and clarified directly in terms of the
integrability properties of the random matrix model.

...
\end{minipage}

\bigskip

\end{figure}

%%%%%%%%%%%%%%%%%%%%%%%%%%%% end of figures %%%%%%%%%%%%%%%%%%%%%%%%%%%%

\section{Abstract}

The abstract should be self-contained. Therefore, do not refer to the
list of references. Instead, quote the reference in full, as follows:
Wettig \& Brown (1996, NewA, 1, 17).

\section{Keywords}
\label{keywd}
\enlargethispage*{2.5pc}

In electronic publications a proper classification is more important
than ever. Elsevier's physics journals use several keyword
schemes:\ 

\begin{description}
\item[Keywords:] Uncontrolled keywords.
\item[PACS:] The PACS scheme, developed and maintained by the AIP,
covers the whole field of Physics.
See \url{http://www.aip.org/pacs/pacs.html}
or \url{http://www.elsevier.com/locate/pacs}.
\item[MSC:] The MSC scheme, developed and maintained by the AMS,
covers the whole field of Mathematics.
See \url{http://www.ams.org/msc}
or \url{http://www.elsevier.com/locate/msc}.
\end{description}

Keywords are entered below the abstract in the following way:
\pagebreak[4]
\begin{verbatim}
\begin{keyword}
Keyword \sep Keyword
\PACS PACS code \sep PACS code
\MSC MSC code \sep MSC code
\end{keyword}
\end{verbatim}

\section{Cross-references}
\label{xrefs}

In electronic publications articles may be internally hyperlinked.
Hyperlinks are generated from proper cross-references in the article.

For example, the words Fig. 1 will never be more than simple text,
whereas the proper cross-reference \verb|\ref{mapfigure}| may be
turned into a hyperlink to the figure itself.

In the same way, the words Ref. [1] will fail to
turn into a hyperlink; the proper cross-reference is 
\verb|\cite{Gea97}|.

Cross-referencing is possible in \LaTeX{} for sections, subsections,
formulae, figures, tables, and literature references.

\section{PostScript figures}
\label{psfigs}

\LaTeX{} and PostScript have had a long and successful relationship.
There are three packages for including PostScript figures:
\begin{itemize}
\item \texttt{graphics}.
This simple package provides the command\\
\verb|\includegraphics*[<llx,lly>][<urx,ury>]{file}|.
The \texttt{*} is optional; it enables the PostScript feature of clipping.
In its simplest form,\\
\verb|\includegraphics{file}|,
it includes the figure in the PostScript file \texttt{file}
without resizing.
\item \texttt{graphicx}.
This package provides the command\\
\verb|\includegraphics*[key--value list]{file}|.
The \texttt{*} is optional; it enables the PostScript feature of clipping.
Often used keys are:
\def\labelitemii{--}
\begin{itemize}
\item \texttt{scale=.40} to scale the size of the figure with 40\%,
\item \texttt{width=25pc}, \texttt{height=15pc} to set the width or
height of the figure,
\item \texttt{angle=90} to rotate the figure over $90^\circ$.
\end{itemize}
\item \texttt{epsfig}.
This package is really the \texttt{graphicx} package,
but it allows one to include PostScript figures using 
the familiar commands from 
the earlier packages \texttt{epsfig} and \texttt{psfig}.
\end{itemize}
For detailed information, see the documentation of the \texttt{graphics}
packages, in particular the file \texttt{grfguide.tex}.
\begin{figure}
\leftmargin=2pc
\begin{verbatim}
\begin{figure}
\begin{center}
\includegraphics*[width=5cm]{name.eps}
\end{center}
\caption{An example of a figure.}
\label{fig:exmp}
\end{figure}
\end{verbatim} 
\caption{An example of a figure.}
\label{fig:exmp}
\end{figure}

\section{Mathematical symbols}
\label{symbols}

Many physics authors require more mathematical symbols than the
few that are provided in standard \LaTeX. A useful package for
additional symbols is the \texttt{amssymb} package, developed by the American
Mathematical Society.  This package includes such oft used symbols as
\verb|\lesssim| for $\lesssim$, \verb|\gtrsim| for $\gtrsim$ or
\verb|\hbar| for $\hbar$.  Note that your \TeX{} system should have
the \texttt{msam} and \texttt{msbm} fonts installed.
If you need only a few symbols, such as \verb|\Box| for $\Box$,
you might try the package \texttt{latexsym}.

In the \texttt{elsart} document class vectors are preferably coded as 
\verb|\vec{a}| instead of \verb|\bf{a}| or \verb|\pol{a}|.

\section{Line numbering}
\label{linenumbering}

\begin{linenumbers}
Reviewing an electronic version of an article has many
advantages. However, reviewers have a harder task indicating remarks
and desired changes to the article. Their task is made easier if the
lines of the article are numbered. \LaTeX's \texttt{lineno} package
performs this task. It is compatible with \texttt{elsart}.
\end{linenumbers}

\section{The Bibliography}
\label{thebib}

In \LaTeX{} literature references are listed in the
\verb|thebibliography| environment. Each reference is a
\verb|\bibitem|; each \verb|\bibitem| is identified by a label, by
which it can be cited in the text: \verb|\bibitem{ESG96}| is cited as
\verb|\cite{ESG96}|.

Version 2.16 of \texttt{elsart} introduces the \texttt{subbibitems}
environment. The references in a \texttt{subbibitems} environment
have the same major reference number, and are counted by letters
a, b, etc. The \texttt{subbibitems} environment has
a label of its own: \verb|\begin{subbibitems}{label}|. 
It can therefore be referred to as \verb|\cite{label}|, which produces
a citation like [7a--c]. A short citation like [7] can be produced by
adding \texttt{:s} to the label: \verb|\cite{label:s}|. Example: See
Refs. \cite{Lee}, or in short form, see Refs. \cite{Lee:s}.

Version 2.16 of \texttt{elsart} also introduces the possibility to
insert notes into the bibliography, by using the \verb|\note| command.
In a \texttt{subbibitems} environment it must be the last item.
Example: See Refs. \cite{note,x}.

These options do not work well with the \texttt{natbib} package.

\section{Template article}
\label{templart}

There is a template article \file{templat-num.tex}, which you can use as a
skeleton for your own article. It is available from Elsevier's Author
Gateway, \url{http://authors.elsevier.com/locate/latex}.

\begin{thebibliography}{9}

\bibitem{Lamp86}
Leslie Lamport: \LaTeX, {\em A document preparation system},
2nd edition, Addison-Wesley (Reading, Massachusetts, 1994).

\bibitem{WB96}
Wettig, T., \& Brown, G.E.,
The evolution of relativistic binary pulsars,
1996, NewA, 1, 17-34.

\bibitem{ESG96}
Elson, R.A.W., Santiago, B.X., \& Gilmore, G.F.,
Halo stars, starbursts, and distant globular clusters:
A survey of unresolved objects in the Hubble Deep Field,
1996, NewA, 1, 1-16.
% (\url{http://www.elsevier.com/PII/S1384107696000061})

\bibitem{Gea97}
Governato, F., Moore, B., Cen, R., Stadel, J., Lake, G., \& Quinn, T.,
The Local Group as a test of cosmological models,
1997, NewA 2, 91-106.

\bibitem{note}%
  \note We consider an exactly solvable two-band model of electrons moving in
one dimension and interacting with a $\delta$-function spin-exchange
potential.

\begin{subbibitems}{Lee}
\bibitem{Lee:a} N.~Nagaosa and P.~A.~Lee, Phys. Rev. Lett. {\bf 79}, 3755
(1997).
\bibitem{Lee:b}C.~P\'epin and P.~A.~Lee, Phys. Rev. Lett. {\bf 81},
  2779 (1998).
\end{subbibitems}

\bibitem{x} K.~Gorny, O.~M.~Vyasilev,
J.~A.~Marindale, V.~A.~Nandor, C.~H.~Pennington, P.~C.~Hammel,
W.~L.~Hults, J.~L.~Smith, P.~L.~Kuhns, A.~P.~Reyes and W.~G.~Moulton,
Phys. Rev. Lett.  {\bf 82}, 177 (1999).
\note These references demonstrate that for some high-$T_c$ compounds
the gap does not seem to depend on the magnetic field.

\end{thebibliography}

\end{document}
